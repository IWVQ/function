
\titleformat{\subsection}[runin]{\large \bfseries}{\thesubsection.}{10pt}{\bfseries}
\titlespacing{\subsection}{0pt}{10pt}{0pt}

\chapter{Representación interna}
   Las expresiones escritas son traducidas a una forma interna para que puedan ser manipuladas y entendidas por el interprete de manera fácil, estas formas internas es una variante de las llamadas ``expresiones lambda puras'' y se representan como arboles.
   
   \section{Una expresion lambda extendida}
      La forma que tienen las expresiones en Function v0.5 están basadas en expresiones lambda, las expresiones lambda son objetos diseñados para describir y estudiar el comportamiento de las funciones, estas expresiones lambda pertenecen a la rama de las matemáticas y la lógica llamada calculo lambda.
      \\
      
      En Function v0.5 todas las expresiones están formadas por expresiones lambda puras a la que se le agregaron algunos elementos como las tuplas, constructores de listas, la captura de fallos, patrones y el tipado de identificadores, estas expresiones están formadas como arboles para garantizar una manipulación sencilla.
      \\
      
      En resumen la expresion básica y nativa en Function v0.5 es una expresion formada unicamente por:
      
      \begin{enumerate}
         \item Números(negativos, infinitos y nan incluido).
         \item Lógicos.
         \item Caracteres.
         \item Lista vacía.
         \item Fallo.
         \item Captura de fallo.
         \item Tuplas.
         \item Constructor de lista.
         \item Abstracción lambda de un solo patrón.
         \item Aplicaciones.
         \item Identificadores.
         \item Anónimo.
         \item Primitivas.
      \end{enumerate}
      
      Y los patrones estan formados unicamente por:
      
      \begin{enumerate}
         \item Números(negativos, infinitos y nan incluido).
         \item Lógicos.
         \item Caracteres.
         \item Lista vacía.
         \item Fallo.
         \item Tuplas.
         \item Constructor de lista.
         \item Identificadores.
         \item Anónimo.
      \end{enumerate}
      
   \section{Convertir valores a cadenas}
      Para convertir cualquier valor a una cadena se puede utilizar las siguientes funciones.
      
      \subsection*{ValueToStr}: \texttt{ValueToStr~<argumento>}\\
      Esta función convierte cualquier valor en una cadena en una forma estilizada.
      
      \begin{fxcode}
         \arrowcode{ValueToStr (\textbackslash x ->~x)}\\
         \outcode{\textquotedbl(\textbackslash \textbackslash ~x ->~x)\textquotedbl}\\
         \arrowcode{ValueToStr (2, [0, 4, \textquotesingle e\textquotesingle])}\\
         \outcode{\textquotedbl(2, [0, 4, \textbackslash\textquotesingle e\textbackslash\textquotesingle])\textquotedbl}\\
         \arrowcode{ValueToStr (\textbackslash(y: real) ->~y\^{}3 + y\^{}2 + y + 1)}\\
         \outcode{\textquotedbl(\textbackslash\textbackslash (y : real) ->~((y \^{} 3 + y \^{} 2) + y) + 1)\textquotedbl}\\
         \arrowcode{Print(ValueToStr (\textbackslash x y -> ~x + y))}\\
         \outcode{(\textbackslash~ x y ->~x + y)}\\
         \outcode{()}
      \end{fxcode}

      \subsection*{ValueToStrFull}: \texttt{ValueToStrFull~<argumento>}\\
      Esta función convierte un valor en cadena mostrando su forma interna tal como es.
      
      \begin{fxcode}
         \arrowcode{ValueToStrFull (\textbackslash x ->~x)}\\
         \outcode{\textquotedbl(\textbackslash\textbackslash~ x ->~x)\textquotedbl}\\
         \arrowcode{ValueToStrFull (2, [0, 4, \textquotesingle e\textquotesingle])}\\
         \outcode{\textquotedbl(2,0~>| 4~>| \textbackslash\textquotesingle e\textbackslash\textquotesingle~>| [])\textquotedbl}\\
         \arrowcode{ValueToStrFull (\textbackslash(y: real) ->~y\^{}3 + y\^{}2 + y + 1)}\\
         \outcode{\textquotedbl(\textbackslash\textbackslash (y : real) ->~(+) ((+) ((+) ((\^{}) (y,3),(\^{}) (y,2)),y),1))\textquotedbl}\\
         \arrowcode{Print(ValueToStrFull (\textbackslash x y ->~x + y))}\\
         \outcode{(\textbackslash~ x ->~(\textbackslash~ y ->~(+) (x,y)))}\\
         \outcode{()}
      \end{fxcode}
      
      \subsection*{TypeToStr}: \texttt{TypeToStr~<argumento>}\\
      Esta función lo que hace es inferir el tipo de dato de un valor y devolverla en forma de cadena.
      
      \begin{fxcode}
         \arrowcode{TypeToStr (\textbackslash x ->~x)}\\
         \outcode{\textquotedbl \_ ->~\_\textquotedbl}\\
         \arrowcode{TypeToStr (\textbackslash(x: int) ->~\textquotesingle a\textquotesingle)}\\
         \outcode{\textquotedbl int ->~char\textquotedbl}\\
         \arrowcode{TypeToStr (v: ([int], real) ->~char)}\\
         \outcode{\textquotedbl([int], real) ->~char\textquotedbl}\\
         \arrowcode{TypeToStr ([1, 2, 3] ++ [4, 5], [\textquotesingle a\textquotesingle, \textquotesingle b\textquotesingle], [2, \textquotesingle a\textquotesingle])}\\
         \outcode{\textquotedbl([nat], [char], [\_])\textquotedbl}
      \end{fxcode}
      
