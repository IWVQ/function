
\titleformat{\subsection}[runin]{\large \bfseries}{\thesubsection.}{10pt}{\bfseries}
\titlespacing{\subsection}{0pt}{10pt}{0pt}

\chapter{Tuplas}
   Una tupla es una secuencia finita de cualquier valor encerrado entre paréntesis y separados por comas.
   \\
   
   La cantidad de elementos de la tupla también llamado aridad puede ser cero o mas pero no 1 pues habría un conflicto con el uso de los paréntesis para asociar y agrupar expresiones, en caso de que la aridad sea 0 se le llama tupla vacía o trivial.
   
   \begin{fxcode}
      \arrowcode{(2, 3, false, \textquotesingle b\textquotesingle)}\\
      \outcode{(2, 3, false, \textquotesingle b\textquotesingle)}\\
      \arrowcode{()}\\
      \outcode{()}\\
      \arrowcode{(true || false, 3 + 4, \textquotedbl hola \textquotedbl~++ \textquotedbl mundo\textquotedbl)}\\
      \outcode{(true, 7, \textquotedbl hola mundo\textquotedbl)}
   \end{fxcode}
   
   Las tuplas tienen su origen en las n-tuplas matemáticas y representan eso, son usados como pares ordenados, n-tuplas o como los argumentos de funciones.
   
   \section{Funciones sobre tuplas}
      \subsection*{Aridad}: \texttt{Arity <tupla>}\\
      Donde \texttt{<tupla>} debe ser una tupla, esta función devuelve la aridad de la \texttt{<tupla>}.
         
      \begin{fxcode}
         \arrowcode{Arity(1, \textquotesingle2\textquotesingle, true, 43)}\\
         \outcode{4}
      \end{fxcode}
      
      \subsection*{Tupla trivial}: \texttt{Trivial <tupla>}\\
      Donde \texttt{<tupla>} es una tupla, esta función verifica si la tupla es vacía o no.
      
      \begin{fxcode}
         \arrowcode{Trivial ()}\\
         \outcode{true}\\
         \arrowcode{Trivial (4, 5)}\\
         \outcode{false}
      \end{fxcode}
   
   \section{Elementos en la tupla}
      \subsection*{Elemento i-ésimo} \texttt{<tupla>\{<indice>\}}\\
      Donde \texttt{<tupla>} es una tupla e \texttt{<indice>} es un número que indica un indice, esta función obtiene el \texttt{<indice>}-ésimo elemento de la tupla \texttt{<tupla>}.
      
      \begin{fxcode}
         \arrowcode{(\textquotedbl hola\textquotedbl, \textquotedbl mundo\textquotedbl, 1, 2)\{0\}}\\
         \outcode{\textquotedbl hola\textquotedbl}
      \end{fxcode}
      
      El indice de una tupla se cuenta en base a cero, es decir el primer elemento sera el 0-ésimo elemento, el segundo el 1-ésimo elemento y así sucesivamente
      
      \subsection*{Notación multi-indice} \texttt{<tupla>\{<indice>, ..., <indice>\}}\\
      Dentro de las llaves se puede colocar cualquier cantidad de indices y obtienen el \texttt{(<indice>, ..., <indice>)}-ésimo elemento de la tupla.
      
      \begin{fxcode}
         \arrowcode{((1, 2, 3), (4, 5, 6), (7, 8, 9))\{1, 2\}}\\
         \outcode{6}
      \end{fxcode}
      
      Si la cantidad de indices es cero entonces devolverá el mismo valor.
      
      \begin{fxcode}
         \arrowcode{((1, 2, 3), (4, 5, 6), (7, 8, 9))\{\}}\\
         \outcode{((1, 2, 3), (4, 5, 6), (7, 8, 9))}\\
         \arrowcode{12\{\}}\\
         \outcode{12}
      \end{fxcode}
      
      Los indices siempre tienen que ser números naturales.
      
   \section{Pares ordenados}
      Los pares ordenados son tuplas de dos elementos, como en toda tupla los elementos pueden ser cualquier valor.
      
      \subsection*{Comprobar si una tupla es par}: \texttt{IsPair <tupla>}\\
      Donde \texttt{<tupla>} es una tupla, esta función verifica si la tupla es un par ordenado.
      
      \begin{fxcode}
         \arrowcode{IsPair(3, \textquotesingle4\textquotesingle)}\\
         \outcode{true}
      \end{fxcode}
      
      \subsection*{Abscisa}: \texttt{PairX <par>}\\
      Donde \texttt{<par>} es un par ordenado, obtiene el primer elemento del par.
      
      \begin{fxcode}
         \arrowcode{PairX(3, \textquotesingle4\textquotesingle)}\\
         \outcode{3}
      \end{fxcode}
      
      \subsection*{Ordenada}: \texttt{PairY <par>}\\
      Donde \texttt{<par>} es un par ordenado, obtiene el segundo elemento del par.
      
      \begin{fxcode}
         \arrowcode{PairY(3, \textquotesingle4\textquotesingle)}\\
         \outcode{\textquotesingle4\textquotesingle}
      \end{fxcode}
      
      \subsection*{Igualdad}: \texttt{<par>~=~<par>}\\
      Donde \texttt{<par>} son pares ordenados, verifica si dos pares ordenados son iguales.
      
      \begin{fxcode}
         \arrowcode{(3, 4) = (3, \textquotesingle4\textquotesingle)}\\
         \outcode{false}
      \end{fxcode}
      
      \subsection*{Desigualdad}: \texttt{<par>~<>~<par>}\\
      Donde \texttt{<par>} son pares ordenados, verifica si dos pares ordenados son diferentes.
      
      \begin{fxcode}
         \arrowcode{(3, 4) <>~(3, \textquotesingle4\textquotesingle)}\\
         \outcode{true}
      \end{fxcode}
      
   
   
   