
\titleformat{\subsection}[runin]{\large \bfseries}{\thesubsection.}{10pt}{\bfseries}
\titlespacing{\subsection}{0pt}{10pt}{0pt}

\chapter{Control del sistema}
   Las funciones presentadas aquí son funciones que ademas de devolver algo realizan una acción para controlar el sistema.
   
   \section{Salir del programa}
      \texttt{Quit()}
      \\
      
      Devuelve una tupla vacía y cierra el software.
      
      \begin{fxcode}
         \arrowcode{Quit()}
      \end{fxcode}
      
   \section{Interrumpir una tarea}
      \texttt{Interrupt()}
      \\
   
      Devuelve una tupla vacía y aborta la ejecución de una tarea.
   
      \begin{fxcode}
         \arrowcode{Interrupt()}\\
         \outcode{\{Interrupted!\}}
      \end{fxcode}
   
   \section{Reiniciar el sistema}
      \texttt{Restart()}
      \\
      
      Devuelve una tupla vacía y reinicia la consola.
      
      \begin{fxcode}
         \arrowcode{Restart()}\\
         \outcode{---- SHELL RESTARTED ----}
      \end{fxcode}
      
   \section{Resultado anterior}
      \texttt{Ans()}
      \\
      
      Devuelve el resultado de la evaluación anterior.
      
      \begin{fxcode}
         \arrowcode{2 + 3}\\
         \outcode{5}\\
         \arrowcode{Ans()}\\
         \outcode{5}
      \end{fxcode}
      
   \section{Ayuda}
      \texttt{Ans()}
      \\
      
      Despliega el entorno de ayuda y devuelve una tupla vacía.
      
      \begin{fxcode}
         \arrowcode{Help()}\\
         \outcode{} \codecomment{aquí despliega el entorno}
      \end{fxcode}
      
   
   
   
   
   
   
   
   
   
   